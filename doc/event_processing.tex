\documentclass[a4paper, 11pt]{article}
\usepackage{amsmath}
\usepackage{amsfonts}

\title{Realization of the main eGFRD algorithm in the current python code}
\author{Laurens Bossen}

\begin{document}
\maketitle

A number of independent processes or 'coordinates' take place inside a Domain. Independent processes are:
-Diffusion in a number of dimensions
-Decay reactions

When one of the processes in the domain has produced the current event, then a fire_domain is executed. However, when an other
event external to the domain is the current event but the processes need to be propagated anyhow (a burst), then a burst_domain
is executed.
Fire happens when the domain produced the top event of the scheduler. A burst event happens when the event associated with the
domain is still in the scheduler.
Fire->Event is no longer in scheduler
Burst->Event is is still in scheduler


Fire_Single
-process event causing coordinate (a coordinate is an independent process in a domain, eg diffusion, decay)
-process all other coordinates

Burst_Single
-process all coordinates (event was external)

After a fire or burst the domain is 'reduced' to a default NonInteractionSingle and a new domain (Pair, Interaction,
NonInteractionSingle) should be made.

Then reschedule the event


Note that from a Pair, Interaction (Multi) after a Fire or Burst you always 'go back' to a NonInteractionSingle. This means
that you change domain.
A speedup trick for this is to 'save' the original single in the Pair or Interaction domain so that we can potentially reuse
them when a fire or burst event takes place


Processing the event means:
Test if Single && zero-shell && D != 0 then
  skip (the single was a special single->zero shell)
if diffusion == true (D != 0)
  determine new position(s) of particle(s)
   -check also if still in shell
if event is (not Burst, not Escape, Single_Reaction, IV_Reaction, IV_Interaction)
  get new identity for the resulting particle
  calculate new position for particles (for example Interaction, Decay reactions)
apply boundary conditions to new position(s)
actually move/remove/add the particle (update_world)

determine if we need to make a different domain (for example when bursting an Interaction)
draw new event time
reschedule event and domain



Create a Protective Domain:
-generate identifiers for domain and shell
-make domain object (instantiate with particle, shell dimensions, structure/surface, reaction_rules, misc info)
 -make appropriate shell (done by constructor or later in case of Multi)
-place shell in container (move_shell)
-
\end{document}
