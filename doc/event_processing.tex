\documentclass[a4paper, 11pt]{article}
\usepackage{amsmath}
\usepackage{amsfonts}

\title{Realization of the main eGFRD algorithm in the current python code}
\author{Laurens Bossen}

\begin{document}
\maketitle

A number of independent processes or 'coordinates' take place inside a Domain. Independent processes are:
-Diffusion in a number of dimensions
-Decay reactions

When one of the processes in the domain has produced the current event, then a fire\_domain is executed. However, when an other
event external to the domain is the current event but the processes need to be propagated anyhow (a burst), then a burst\_domain
is executed.
Fire happens when the domain produced the top event of the scheduler. A burst event happens when the event associated with the
domain is still in the scheduler.
Fire->Event is no longer in scheduler
Burst->Event is is still in scheduler


Fire\_Single
-process event causing coordinate (a coordinate is an independent process in a domain, eg diffusion, decay)
-process all other coordinates

Burst\_Single
-process all coordinates (event was external)

After a fire or burst the domain is 'reduced' to a default NonInteractionSingle and a new domain (Pair, Interaction,
NonInteractionSingle) should be made.

Then reschedule the event


Note that from a Pair, Interaction (Multi) after a Fire or Burst you always 'go back' to a NonInteractionSingle. This means
that you change domain.
A speedup trick for this is to 'save' the original single in the Pair or Interaction domain so that we can potentially reuse
them when a fire or burst event takes place


Processing the event means:
if Single && zero-shell && D != 0 then
  the single was a 'special' single (zero shell) -> make\_new\_domain
else
  the single was a 'normal' domain and event should be processed

  (calculate new position(s))
  if (D != 0) && dt>0 (time has passed)
    (there was actual relevant diffusion)
    determine new position(s) of particle(s)
    apply boundary conditions
    -check if still in shell
  else
    (no diffusion has taken place)
    positions are old position(s)

  (determine identitie(s))
  if event is Single\_Reaction, IV\_Reaction, IV\_Interaction
    get new identity for the resulting particle
    calculate new position for particles (for example Interaction, Decay reactions)
    apply boundary conditions to new position(s)
    actually move/remove/add the particle (update\_world)

  else
    (event was Burst or one of the possible Escape events)
    positions are 'old' positions
    particle identities are old identities
    particle number is old number

    apply boundary conditions to position(s)
    move particles to new positions (make change in the world)

  (we now have n particles in the world with certain identity and position)
  (In principle all particles should still be in the domain)

  remove old domain 
  for all particles
    create default NonInteractionSingle for particle
    schedule domain
    log newly created NonInteractionSingle



    (we now have n particles with certain identity and position)
    move particles
    NonInteractionSingle(s) = cached NonInteractionSingle(s)
    remove old domain
    re-initialize time and size of NonInteractionSingle(s)
    add 




Note that when a Pair or Interaction is made, the shells and domain\_id are removed from the shell container and
domain hash respectively. This doesn't mean that the objects too have gone. The NonInteractionSingles are subsequently
cached in the made domain (Pair, Interaction, Multi?). When the made domain is now fired --the event produced by the
domain is processed-- and the resulting particle has not changed its identity, then the cached domain can be used. The
shell of the cached domain will recreated in the shell container and its domain\_id will be added to the domains hash.



Create a Protective Domain:
-generate identifiers for domain and shell
-make domain object (instantiate with particle, shell dimensions, structure/surface, reaction\_rules, misc info)
 -make appropriate shell (done by constructor or later in case of Multi)
-place shell in container (move\_shell)
-
\end{document}
